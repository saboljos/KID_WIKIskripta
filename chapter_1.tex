\section{Kinetika a dynamika nulového reaktoru}

\textbf{Kinetika reaktoru} = zkoumá časové chování reaktoru se změnou vstupních parametrů.\\

\textbf{Vstupní parametry} = chápeme primárně $k_{\text{ef}}$, resp. $\rho$, a lze je ovlivnit změnou materiálů či geometrií systému.\\

\textbf{Výstupní parametry} = to, co v systému měříme ($P(t)$, $\Phi (\textbf{r}, t) $ atd.).\\

\textbf{Nulový reaktor} = neboli reaktor nulového výkonu; reaktor pracující v takovém výkonovém rozsahu, že jsou jeho zpětné vazby (ZV) zanedbatelné.

\begin{itemize}
  \item Výzkumné a energetické reaktory sem řadit nelze, jelikož se ZV projevují.
  \item Často složité odlišit, u některých nulových reaktorů lze pozorovat ZV (ve vyšších energetických hladinách) a naopak některé energetické reaktory lze provozovat bez ZV (při minimálním provozním výkonu).
\end{itemize}

\textbf{Zpětná vazba} = proces, díky kterému se změna výstupních parametrů ($P$, $\Phi$) může podílet na změnu vstupních parametrů.\\

\textbf{Dynamika reaktoru} = to samé co kinetika, pouze už uvažuje zapojení ZV.

\subsection{Rovnice kinetiky reaktoru}

= Rovnice popisující závislost změny výstupních parametrů (výsledků) na změně vstupních parametrů.\\

K popisu lze využít transportní rovnici, resp. zjednodušenou difúzní rovnici $\rightarrow$ vede na komplikované soustavy, které nelze v obecném případě řešit analyticky (s projevem heterogenity systému).\\

Řešením jsou \textbf{Rovnice bodové kinetiky}, které zanedbávají změnu prostorového rozložení $\rightarrow$ nastane-li změna na vstupních parametrech (zvětší-li se reaktivita), tak změna výstupních parametrů (např. $\Phi$) se ve všech místech změní stějnou měrou $\rightarrow$ výstupní parametry se tedy pouze škálují a průběh zůstává zachován.\\

\textbf{Rovnice jednobodové kinetiky} = kromě prostorové závislosti se zanedbává i energetické rozdělení $\rightarrow$ vede na 1G rovnice.\\

V reálu to tak není, ale kupodivu dávají rovnice přijatelné výsledky.

\subsection{Odvození rovnic jednobodové kinetiky}

Pro odvození se vychází z 1G difúzní rovnice (s konstantním $D$ a $\Sigma_a$):

\begin{equation}
  \boxed{
  \dfrac{\partial n(\textbf{r}, t)}{\partial t} = D \Delta \Phi (\textbf{r}, t) - \Sigma_a \Phi (\textbf{r}, t) + Q (\textbf{r}, t),
  \label{difuzka}}
\end{equation}

kde:

\begin{itemize}
  \item $n$ (1/cm$^3$) značí hustotu neutronů,
  \item $D$ (cm) značí difúzní koeficient,
  \item $\Phi$ (1/cm$^2$s) značí hustotu toku neutronů,
  \item $\Sigma_a$ (1/cm) značí makroskopický účinný průřez pro absorbci a
  \item $Q$ (1/cm$^3$s) značí zdroj neutronů.
\end{itemize}

\subsubsection{Odvození bez vlivu zpožděných neutronů}

Uvažuji zjednodušení tvaru:

$$ Q (\textbf{r}, t) = k_\infty \Sigma_a \Phi (\textbf{r}, t), $$

$$ L^2 = \dfrac{D}{\Sigma_a}, $$

$$ B_m^2 = \dfrac{k_\infty - 1}{L^2}, $$

$$ n(\textbf{r}, t) = \dfrac{\Phi (\textbf{r}, t)}{v}$$

kde:

\begin{itemize}
  \item $k_\infty$ (-) značí koeficient násobenní pro nekonečný systém,
  \item $L$ (cm) značí difúzní délku (po umocnění difúzní plochu),
  \item $B_m$ (1/cm) značí materiálový faktor a
  \item $v$ (cm/s) značí rychlost neutronů (je konstantní, jelikož máme 1G přiblížení)
\end{itemize}

a předpokládáme, že rovnici \eqref{difuzka} lze řešit metodou separace proměnných, tedy:

$$ \Phi (\textbf{r}, t) = \Psi (\textbf{r}) \cdot T(t). $$

Poté rovnice \eqref{difuzka} vede na rovnici:

\begin{equation}
  v D \left ( \dfrac{\Delta \Psi (\textbf{r})}{\Psi (\textbf{r})} + B_m^2 \right ) = \dfrac{1}{T(t)} \dfrac{d T(t)}{d t} = \text{konst.} = - \omega,
  \label{difuzka_v_separaci}
\end{equation}

tedy na 2 obyčejné diferenciální rovnice provázané konstantou $\omega$.\\

Rovnice s $\Psi (\textbf{r})$ vede po uvažování okrajových podmínek (extrapolované rozhraní, konečnost, spojitost apod.) na vlastní funkce, jejichž tvar závisí na použité geometrii a tvaru Laplaciánu (kombinace goniometrických, Besselových, hyperbolických apod.). Řešení vyplývá z jednoduché vlnové rovnice:

$$ \Delta \Psi (\textbf{r}) + B_n^2 \Psi (\textbf{r}) = 0, $$

kde vztah mezi \textbf{vlastními čísly} $B_n$ a \textbf{materiálovým faktorem} $B_m$ je svázán pomocí určené konstanty $\omega$ jako:

$$ \omega_n = vD \cdot (B_n^2-B_m^2). $$

Obecně lze výsledek zapsat tvarem:

$$ \Psi (\textbf{r}) = \sum_n A_n \Psi_n (\textbf{r}), $$

kde $A_n$ značí normalizační konstantu a zjistíme ji z výkonu reaktoru.\\

Rovnice s $T(t)$ vede na exponenciálu tvaru:

$$ T(t) = C e^{- \omega t}. $$

Jelikož je ovšem $\omega$ závislá na volbě vlastních čísel, tak i zde platí superpozice a celkovou hustotu toku neutronů $\Phi (\textbf{r}, t)$ spočteme přes sumu všech vlastních funkcí jako:

\begin{equation}
  \Phi (\textbf{r}, t) = \sum_n A_n \Psi_n (\textbf{r}) e^{- \omega_n t}.
  \label{difuzka_reseni}
\end{equation}

Tabulka \ref{table_vlastni_funkce} udává vlastní čísla a vlastní funkce pro různé geometrie reaktoru (viz ZAF2).

\begin{table}[H]
\centering
\caption{Vlastní čísla a vlastní funkce pro různé geometrie.}
\label{table_vlastni_funkce}
\begin{tabular}{@{}rcc@{}}
\toprule
\textbf{Geometrie}   & $B_n$ (1/cm)         & $\Psi_n$ (-)                               \\ \midrule
\textbf{Nek. deska}  & $n \left ( \dfrac{\pi}{a} \right )$   & $\cos(B_nx)$              \\ [15pt]
\textbf{Nek. válec}  & $n \left ( \dfrac{2,405}{R} \right )$ & $J_0(B_nr)$               \\ [15pt]
\textbf{Koule}       & $n \left ( \dfrac{\pi}{R} \right )$   & $\dfrac{\sin(B_nr)}{r}$   \\ \bottomrule
\end{tabular}
\end{table}

Jelikož vlastní čísla splňují bilanci $B_1 < B_2 < B_3 < ...$, platí to samé i pro $\omega_1 < \omega_2 < \omega_3 < ...$ a první vlastní číslo po chvíli převáží ta zbylá. Proto dále zavádíme \textbf{geometrický faktor} $B_g$ jako první nejmenší vlastní číslo, tedy $B_g = B_1$.\\

Pro stacionární systém navíc platí $\omega = 0$ a poté $B_m = B_g$ (viz ZAF2).\\

Nyní přejdeme k prostorové nezávislosti (což je vlastně smysl celé kapitoly :D). Lze uvažovat (za předpokladu převážení prvního členu v rovnici \eqref{difuzka_reseni}), že:

$$ \Phi (\textbf{r}, t) \doteq v n(t) \Psi_1 (\textbf{r}) $$

a hustota neutronů $n(t)$ je zároveň úměrná maximální hustotě toku v soustavě (předpoklad rovnice jednobodové kinetiky), tedy:

$$ n(t) \doteq \text{konst.} \cdot \Phi_{max} (t). $$

Po dosazení do rovnice \eqref{difuzka_v_separaci} získáme novou rovnici tvaru:

\begin{equation}
  v D \left ( \dfrac{\Delta \Psi_1 (\textbf{r})}{\Psi_1 (\textbf{r})} + B_m^2 \right ) = \dfrac{1}{n(t)} \dfrac{d n(t)}{d t} = \text{konst.} = - \omega_1,
  \label{rovnice_kinetiky_v_separaci}
\end{equation}

která opět vede na 2 obyčejné diferenciální rovnice provázané konstantou $\omega_1$. Nyní už ovšem nejde o superpozici, jelikož uvažujeme pouze první člen (ačkoliv nestacionaritu zachováváme).\\

Pro zopakování a osvěžení paměti, stále platí:

$$ B_g = B_1, $$

$$ \omega_1 = vD \cdot (B_g^2 - B_m^2). $$

Zavedeme novou veličinu $\ell$ (s) jako \textbf{střední dobu života neutronů} vztahem:

\begin{equation}
  \boxed{
  \ell \equiv \dfrac{1}{v \Sigma_a} \dfrac{1}{1+L^2B_g^2}
  \label{stredni_doba_zivota}}
\end{equation}

a připomeneme si 1G rovnici pro stacionární reaktor:

$$ k_{\text{ef}} = \dfrac{k_{\infty}}{1 + L^2 B_g^2}. $$

Z těchto dvou vztahů lze vyjádřit parametr $\omega_1$ (důkaz dosazením) jako:

$$ \omega_1 = -\dfrac{k_{\text{ef}} - 1}{\ell}, $$

Což lze dosadit do rovnice \eqref{rovnice_kinetiky_v_separaci} (část s $\Psi$ už nemusím řešit) a získáváme \textbf{Rovnici jednobodové kinetiky}:

\begin{equation}
  \boxed{
  \dfrac{dn(t)}{dt} = \dfrac{k_{\text{ef}} - 1}{\ell} n(t).
  \label{rovnice_kinetiky_reseni}}
\end{equation}

Tím jsme si odvodili obyčejnou diferenciální rovnici 1. řádu pro hustotu neutronů $n(t)$, kterou lze řešit jednoduše pomocí integračního faktoru/separace proměnných (čímkoliv). Často nás ale více než hustota neutronů zajímá časový vývoj výkonu, tedy $P(t)$. Zde platí jednoduchá úměra:

$$ n(t) \sim P(t) $$

a tedy po přenormování platí:

$$ \dfrac{dP(t)}{dt} = \dfrac{k_{\text{ef}} - 1}{\ell} P(t). $$

S předpokladem počáteční podmínky $P(0) = P_0$ a úvahy, že $k_{\text{ef}} = \text{konst.}$, poté rovnice jednobodové kinetiky pro výkon dává řešení tvaru:

\begin{equation}
  P(t) = P_0 \exp \left ( \dfrac{k_{\text{ef}} - 1}{\ell} t \right ).
  \label{rovnice_kinetiky_vykon}
\end{equation}

\small

\textbf{Př. 1:}

Rovnice \eqref{rovnice_kinetiky_vykon} udává, jak rychle se mění výkon v systému v závislosti na $k_{\text{ef}}$ a $\ell$. Zadefinujeme si \textbf{periodu reaktoru} $T_e$ (s) jako dobu, za kterou se výkon v systému změní e-krát, pomocí vztahu:

\begin{equation}
  \boxed{
  T_e \equiv \dfrac{\ell}{k_{\text{ef}} - 1}.
  \label{perioda}}
\end{equation}

Zatímco $k_{\text{ef}}$ lze ovlivnit (geometrie, obohacení, materiály), $\ell$ je pevně dáno a spjato se systémem\footnote{Teoreticky to lze také ovlivnit, ale asi těžko z rychlého reaktoru uělám tak jednoduše lehkovodní, žejo.}. Přehled rozsahů pro různé systémy zobrazuje tabulka \ref{table_stredni_doby_zivota}. Je tedy vidět, že např. rychlý reaktor bude na změny $k_{\text{ef}}$ reagovat mnohem rychleji, než reaktor moderovaný grafitem.\\

\begin{table}[H]
\small
\centering
\caption{\small Střední doby života pro různé typy reaktorů.}
\label{table_stredni_doby_zivota}
\begin{tabular}{@{}rc@{}}
\toprule
\textbf{Typ systému} & $\ell$ (s)           \\ \midrule
\textbf{FR}          & $10^{-7}$            \\
\textbf{LWR}         & $10^{-5} - 10^{-4}$  \\
\textbf{Grafit}      & $10^{-3}$            \\ \bottomrule
\end{tabular}
\end{table}

Pokud uvažujeme LWR reaktor ($\ell = 10^{-5}$), tak pro:

\begin{itemize}
  \item $k_{\text{ef}} = 1,01$ vychází perioda $T_e = 0,01$~s a za 1~s se změní výkon $2,69 \cdot 10^{43}$x,
  \item $k_{\text{ef}} = 1,001$ vychází perioda $T_e = 0,1$~s a za 1~s se změní výkon $2,20 \cdot 10^{4}$x,
  \item $k_{\text{ef}} = 1,0001$ vychází perioda $T_e = 1$~s a za 1~s se změní výkon $2,72$x.
\end{itemize}

\normalsize

K rovnici \eqref{rovnice_kinetiky_reseni} je možné dojít i jednoduchou úvahou. Jelikož platí úměra mezi $n(t)$ a $N(t)$:

$$ n(t) \sim N(t), $$

lze vycházet právě z počtu neutronů v jedné generaci. Pro přírůstek mezi generacemi totiž platí:

$$ dN = k_{\text{ef}}N - N, $$

což po vydělení časem $dt$ na LS rovnice, resp. dobou života jedné generace $\ell$ na PS rovnice spěje k tíženému řešení:

$$ \dfrac{dN(t)}{dt} = \dfrac{k_{\text{ef}} - 1}{\ell} N(t). $$

Dále je možné rovnici \eqref{rovnice_kinetiky_reseni} přepsat pomocí reaktivity $\rho$. K tomu si zavedeme \textbf{střední dobu vzniku neutronů} $\Lambda$ (s) jako:

\begin{equation}
  \boxed{
  \Lambda \equiv \dfrac{\ell}{k_{\text{ef}}}.
  \label{stredni_doba_vzniku}}
\end{equation}

Po lehké úpravě, usměrnění rovnice \eqref{rovnice_kinetiky_reseni} a úvaze, že $\Lambda = \text{konst.}$, dostáváme nový výraz pro rovnici jednobodové kinetiky:

\begin{equation}
  \boxed{
  \dfrac{dn(t)}{dt} = \dfrac{\rho (t)}{\Lambda} n(t).
  \label{rovnice_kinetiky_reaktivita}}
\end{equation}

$\Lambda$ v podstatě vyjadřuje dobu, za kterou se zreprodukuje 1 neutron. Platí tedy:

\begin{itemize}
  \item $k_{\text{ef}} > 1$ $\rightarrow$ $\Lambda < \ell$ $\rightarrow$ nadkritický systém a tedy neutrony se zreprodukují rychleji, než je doba jejich života,
  \item $k_{\text{ef}} < 1$ $\rightarrow$ $\Lambda > \ell$ $\rightarrow$ podkritický systém a zreprodukování neutronu trvá déle, než doba jejich života.
\end{itemize}

\subsubsection{Odvození s vlivem zpožděných neutronů}

Nejprve si ujasníme, o co se jedná. Neutrony vznikající při štěpení můžeme členit na:

\begin{itemize}
  \item \textbf{Okamžité neutrony} -- vznikají ihned (do $10^{-13}$~s) emisí z mateřského jádra (FP -- Fission Product) se střední energií cca 2~MeV. Při štepení se FP nacházejí v excitovaném stavu a s přebytkem neutronů $\rightarrow$ těch se mohou zbavit buď za pomoci $\beta^-$ rozpadu (vzniká dceřiné jádro), nebo emisí okamžitého neutronu. Často se tyto FP označují jako \textbf{prekurzory}.
  \item \textbf{Zpožděné neutrony} -- jedná se o neutrony, které se uvolňují až po nějaké době, se stření energií cca 0,5~MeV. Vznikají emisí neutronů z dceřiných jader (DP -- Daughter Product), které vznikají radioaktivním rozpadem FP. DP se často označují jako \textbf{emitory}.
\end{itemize}

\begin{figure}[H]
  \centering
  \includegraphics[width=0.8\textwidth]{img/1.png}
  \caption{Vznik okamžitých a zpožděných neutronů.}
  \label{fig_zpozdenky}
\end{figure}

Ačkoliv jsou zpožděné neutrony emitovány emitory, pro jejich charakteristiku je přiřazujeme původním prekurzorům. Těch mohou být desítky, proto je dělíme do několika skupin (JEFF 8~skupin, ENDF/B 6~skupin) podle poločasu rozpadu. Pro popis se zavádí tzv. \textbf{podíl zpožděných neutronů} $\beta$ (-) jako:

\begin{equation}
  \boxed{
  \beta \equiv \dfrac{\nu_D}{\nu_T},
  \label{zpozdenky}}
\end{equation}

kde:

\begin{itemize}
  \item $\nu_D$ (-) značí střední počet zpožděných neutronů vzniklých při jednom štěpení a
  \item $\nu_T$ (-) značí střední počet všech vzniklých neutronů.
\end{itemize}

Obdobně lze zavést $\nu_i$, $\beta_i$, $T_{1/2}^i$, $\tau_i$ a $\lambda_i$ pro jednotlivé skupiny (rodiny) zpožděných neutronů.\\

Dále se pro popis zavádí tzv. \textbf{efektivní střední doba života} $\ell^*$ (s) jako:

\begin{equation}
  \boxed{
  \ell^* \equiv \ell(1-\beta) + \sum_i \beta_i \tau_i,
  \label{efektivni_stredni_doba_zivota}}
\end{equation}

\textbf{efektivní střední doba vzniku} $\Lambda^*$ (s) jako:

\begin{equation}
  \boxed{
  \Lambda^* = \dfrac{\ell^*}{k_{\text{ef}}}
  \label{efektivni_stredni_doba_vzniku}}
\end{equation}

a \textbf{efektivní perioda reaktoru} $T_e^*$ (s) jako:

\begin{equation}
  \boxed{
  T_e^* \equiv \dfrac{\ell^*}{k_{\text{ef}} - 1}.
  \label{efektivni_perioda}}
\end{equation}

V podstatě se jedná o vážený průměr přes koeficienty $\beta$, a ačkoliv je podíl zpožděných neutronů minimální (do 1~\%), díky dlouhým $\tau$ se efektivní doba života velmi prodlouží a perioda reaktoru natáhne. Proto jsou zpožděné neutrony velmi důležité k řízení reaktoru. Nastane-li kritičnost na okamžitých neutronech, tento prodlužovací efekt zcela vymizí, perioda reaktoru se zkrátí až o několik řádů a máme tu druhý Černobyl.\\

\small

\textbf{Př. 2:}

Pro klasický PWR reaktor platí, že $\sum \beta_i \tau_i \approx 0,1$~s. Vezměme hodnoty z př. 1 a koukněme se, jak se změní perioda reaktoru:

\begin{itemize}
  \item $k_{\text{ef}} = 1,01$ vychází efektivní perioda $T_e^* = 10$~s a za 1~s se změní výkon 1,105x (původně $2,69 \cdot 10^{43}$x),
  \item $k_{\text{ef}} = 1,001$ vychází efektivní perioda $T_e^* = 100$~s a za 1~s se změní výkon 1,010x (původně $2,20 \cdot 10^{4}$x),
  \item $k_{\text{ef}} = 1,0001$ vychází efektivní perioda $T_e^* = 1000$~s a za 1~s se změní výkon 1,001x (původně $2,72$x).
\end{itemize}

Je vidět, že s uvažováním zpožděných neutronů se efektivní perioda natáhne o několik řádů a reaktor už není tolik citlivý na změnu $k_{\text{ef}}$.\\

\normalsize

Lépe se řídí takové systémy, které mají větší $\beta$. Ve skutečnosti není reaktor takto ideální. Je třeba dále započítávat fotoneutrony (vznikající ($\gamma$,n) reakcí na lehkých jádrech, např. Be), více skupin zpožděných neutronů apod. Důležitá je i energetická závislost. Jelikož zpožděné neutrony vznikají s menší energií (0,5~Mev vs. 2~Mev) a mají náskok ve zpomalování. Díky nižší energii nemohou zpožděné neutrony nikdy zapříčinit štepení na štěpitelných jádrech.\\

Kvůli tomu všemu se zavádí tzv. \textbf{efektivní podíl zpožděných neutronů} $\beta_{\text{ef}}$ (-), což je umělá hodnota, která koriguje energetický rozdíl ve skupinách, jelikož každá ze skupin má jiný vliv na štepení. Lze ji zavést pomocí vztahu:

\begin{equation}
  \beta_{\text{ef}} = \beta \cdot I,
\end{equation}

kde $I$ (-) značí tzv. \textbf{funkci vlivu} a závisí na konkrétním reaktoru. Říká, jak je snadné pro zpožděné neutrony štěpit, oproti okamžitým neutronům. Obecně se pohybuje okolo $\approx 1$, při bližším studiu lze napsat: FR $<1$ a LWR $>1$.\\

Nyní si ještě ukažme rovnice jednobodové kinetiky se zpožděnými neutrony. Odvození je podobné jako v předcházejícím případě, pouze se původní vztah modifikuje. Výsledkem je soustava lineárních diferenciálních rovnic v destrukčním tvaru:

\begin{equation}
  \boxed{
  \dfrac{dN}{dt} = \dfrac{k_{\text{ef}}(1-\beta_{\text{ef}})-1}{\ell} N(t) + \sum_{i=1}^m \lambda_i C_i(t),
  \label{rovnice_kinetiky_zpozdenky_1}}
\end{equation}

\begin{equation}
  \boxed{
  \dfrac{dC_i}{dt} = -\lambda_i C_i(t) + \dfrac{\beta_{\text{ef},i} k_{\text{ef}} N(t)}{\ell},
  \label{rovnice_kinetiky_zpozdenky_2}}
\end{equation}

resp. rovnice v produkčním tvaru:

\begin{equation}
  \boxed{
  \dfrac{dN}{dt} = \dfrac{\rho - \beta_{\text{ef}}}{\Lambda} N(t) + \sum_{i=1}^m \lambda_i C_i(t),
  \label{rovnice_kinetiky_zpozdenky_3}}
\end{equation}

\begin{equation}
  \boxed{
  \dfrac{dC_i}{dt} = -\lambda_i C_i(t) + \dfrac{\beta_{\text{ef},i}  N(t)}{\Lambda}.
  \label{rovnice_kinetiky_zpozdenky_4}}
\end{equation}

\subsubsection{Přehled vzorečků}

Na závěr kapitolky rychlá vzorečkiáda (viz tabulka \ref{table_vzorecky_kinetika}).

\begin{table}[H]
\centering
\caption{Vzorečky s rovnicemi jednobodové kinetiky.}
\label{table_vzorecky_kinetika}
\begin{tabular}{@{}rcc@{}}
\toprule
\textbf{Parametr}                 & \textbf{Bez zpožděnek} & \textbf{Se zpožděnkami}   \\ \midrule
\textbf{Střední doba života}      & $\ell = \dfrac{1}{v \Sigma_a} \dfrac{1}{1+L^2B_g^2}$    & $\ell^* = \ell(1-\beta) + \sum_i \beta_i \tau_i$      \\ [15pt]
\textbf{Střední doba vzniku}      & $\Lambda = \dfrac{\ell}{k_{\text{ef}}}$                 & $\Lambda^* = \dfrac{\ell^*}{k_{\text{ef}}}$           \\ [15pt]
\textbf{Perioda reaktoru}         & $T_e = \dfrac{\ell}{k_{\text{ef}} - 1}$                 & $T_e^* = \dfrac{\ell^*}{k_{\text{ef}} - 1}$           \\ [15pt]
\textbf{R-ce v destrukčním tvaru} & $\dfrac{dN}{dt} = \dfrac{k_{\text{ef}} - 1}{\ell} N(t)$ & $\dfrac{dN}{dt} = \dfrac{k_{\text{ef}}(1-\beta_{\text{ef}})-1}{\ell} N(t) + \sum_{i=1}^m \lambda_i C_i(t)$        \\ [15pt]
                                  &                                                         & $\dfrac{dC_i}{dt} = -\lambda_i C_i(t)+\dfrac{\beta_{\text{ef},i} k_{\text{ef}} N(t)}{\ell}$           \\ [15pt]
\textbf{R-ce v produkčním tvaru}  & $\dfrac{dN}{dt} = \dfrac{\rho}{\Lambda} N(t)$           & $\dfrac{dN}{dt} = \dfrac{\rho - \beta_{\text{ef}}}{\Lambda} N(t) + \sum_{i=1}^m \lambda_i C_i(t)$                 \\ [15pt]
                                  &                                                         & $\dfrac{dC_i}{dt} = -\lambda_i C_i(t) + \dfrac{\beta_{\text{ef},i}  N(t)}{\Lambda}$                            \\ [15pt] \bottomrule
\end{tabular}
\end{table}

\subsection{Řešení rovnic jednobodové kinetiky}

\subsubsection{Řešení bez vlivu zpožděných neutronů}

Řešení rovnice bez započtení zpožděných neutronů je triviální (jedná se pouze o jednu lineární diferenciální rovnici prvního řádu) a už zde bylo zmíněno. Pro zopakování, řešíme rovnici:

$$ \dfrac{dN}{dt} = \dfrac{\rho (t)}{\Lambda} N(t) $$

s počáteční podmínkou $N(0) = N_0$ (ustálený stav).\\

\textbf{a) Konstatní reaktivita}

Při $\rho = \text{konst.}$ vede rovnice na exponenciální řešení:

\begin{equation}
  \boxed{
  N(t) = N_0 \exp \left ( \dfrac{\rho_0}{\Lambda} t \right ).
  \label{kinetika_reseni}}
\end{equation}

\subsubsection{Řešení s vlivem zpožděných neutronů}

Řešíme soustavu lineárních diferenciálních rovnic:

$$ \dfrac{dN}{dt} = \dfrac{\rho(t) - \beta_{\text{ef}}}{\Lambda} N(t) + \sum_{i=1}^m \lambda_i C_i(t), $$

$$ \dfrac{dC_i}{dt} = -\lambda_i C_i(t) + \dfrac{\beta_{\text{ef},i}  N(t)}{\Lambda}. $$

s počátečními podmínkami (ustálený stav) $N(0) = N_0$ a $C_i(0) = C_{i,0}$. Tento stav lze předpokládat, pokud se reaktor nachází dostatečně dlouhou dobu v ustáleném stavu (při konstantním výkonu), potom je i koncentrace jader $C_i$ konstantní. Tuto koncentraci lze získat z druhé rovnice z doby před $t=0$:

$$ 0 = -\lambda_i C_{i,0} + \dfrac{\beta_{\text{ef},i}  N_0}{\Lambda \lambda_i} \rightarrow C_{i,0} = \dfrac{\beta_{\text{ef},i}  N_0}{\Lambda}. $$

Aplikujme na obě rovnice LT, dostaneme:

$$ s \tilde{N}(s) - N_0 = \dfrac{\mathcal{L}[\rho(t) N(t)](s)}{\Lambda} - \dfrac{\beta_{\text{ef}} \tilde{N}(s)}{\Lambda} + \sum_{i=1}^m \lambda_i \tilde{C_i}(s), $$

$$ s \tilde{C_i}(s) - C_{i,0} = -\lambda_i \tilde{C_i}(s) + \dfrac{\beta_{\text{ef},i}  \tilde{N}(s)}{\Lambda}. $$

Do druhé rovnice můžeme rovnou dosadit za $C_{i,0}$ (známe z počátečních podmínek, $N_0$ zachováváme, to získáme z normalizace výkonu), čímž získáváme soustavu algebraických rovnic:

$$ s \tilde{N}(s) - N_0 = \dfrac{\mathcal{L}[\rho(t) N(t)](s)}{\Lambda} - \dfrac{\beta_{\text{ef}} \tilde{N}(s)}{\Lambda} + \sum_{i=1}^m \lambda_i \tilde{C_i}(s), $$

$$ s \tilde{C_i}(s) - \dfrac{\beta_{\text{ef},i}  N_0}{\Lambda} = -\lambda_i \tilde{C_i}(s) + \dfrac{\beta_{\text{ef},i}  \tilde{N}(s)}{\Lambda}. $$

Pokračujeme tak, že z druhé rovnice vyjádříme $\lambda_i \tilde{C_i}(s)$:

$$ \lambda_i \tilde{C_i}(s) = \dfrac{\beta_{\text{ef},i} \lambda_i \tilde{N}(s) + \beta_{\text{ef},i}N_0}{\Lambda(\lambda_i + s)} $$

a dosadíme do první rovnice (s vynásobením $\Lambda$):

$$ \Lambda (s \tilde{N}(s) - N_0) = \mathcal{L}[\rho(t) N(t)](s) - \beta_{\text{ef}} \tilde{N}(s) + \sum_{i=1}^m \dfrac{\beta_{\text{ef},i} \lambda_i \tilde{N}(s)}{\lambda_i + s} + \sum_{i=1}^m \dfrac{\beta_{\text{ef},i}N_0}{\lambda_i + s}. $$

Nyní rozepíšeme $\beta_{\text{ef}}$ do sumy přes $\beta_{\text{ef},i}$ a rozšíříme přes $\lambda_i + s$. Po vynásobení se něco požere:

$$ \Lambda (s \tilde{N}(s) - N_0) = \mathcal{L}[\rho(t) N(t)](s) - \sum_{i=1}^m \dfrac{\beta_{\text{ef},i} \tilde{N}(s)}{\lambda_i + s}(\lambda_i + s) + \sum_{i=1}^m \dfrac{\beta_{\text{ef},i} \lambda_i \tilde{N}(s)}{\lambda_i + s} + \sum_{i=1}^m \dfrac{\beta_{\text{ef},i}N_0}{\lambda_i + s}, $$

$$ \Lambda (s \tilde{N}(s) - N_0) = \mathcal{L}[\rho(t) N(t)](s) - \sum_{i=1}^m \dfrac{\beta_{\text{ef},i} \lambda_i \tilde{N}(s)}{\lambda_i + s} - \sum_{i=1}^m \dfrac{\beta_{\text{ef},i} s \tilde{N}(s)}{\lambda_i + s} + \sum_{i=1}^m \dfrac{\beta_{\text{ef},i} \lambda_i \tilde{N}(s)}{\lambda_i + s} + \sum_{i=1}^m \dfrac{\beta_{\text{ef},i}N_0}{\lambda_i + s}, $$

$$ \Lambda (s \tilde{N}(s) - N_0) = \mathcal{L}[\rho(t) N(t)](s) - \sum_{i=1}^m \dfrac{\beta_{\text{ef},i} s \tilde{N}(s)}{\lambda_i + s} + \sum_{i=1}^m \dfrac{\beta_{\text{ef},i}N_0}{\lambda_i + s}. $$

Z PS rovnice vytáhneme $(s \tilde{N}(s) - N_0)$:

$$ \Lambda (s \tilde{N}(s) - N_0) = \mathcal{L}[\rho(t) N(t)](s) - \left [ \sum_{i=1}^m \dfrac{\beta_{\text{ef},i}}{\lambda_i + s}(s \tilde{N}(s) - N_0) \right ]$$

a po převedení sumy na LS:

$$ \left ( \Lambda + \sum_{i=1}^m \dfrac{\beta_{\text{ef},i}}{\lambda_i + s} \right ) (s \tilde{N}(s) - N_0) = \mathcal{L}[\rho(t) N(t)](s). $$

Nyní je konečně možné vyjádřit $\tilde{N}(s)$:

$$ \tilde{N}(s) = \dfrac{\mathcal{L}[\rho(t) N(t)](s)}{s \left ( \Lambda + \sum_{i=1}^m \dfrac{\beta_{\text{ef},i}}{\lambda_i + s} \right )} + \dfrac{N_0}{s}. $$

Nyní pro zjednodušení zápisu zavedeme tzv. \textbf{přenosovou funkci nulového reaktoru} $\tilde{G_0}(s)$ vztahem:

\begin{equation}
  \boxed{
  \tilde{G_0}(s) \equiv \dfrac{1}{s \left ( \Lambda + \sum_{i=1}^m \dfrac{\beta_{\text{ef},i}}{\lambda_i + s} \right )}.
  \label{prenosova_funkce}
  }
\end{equation}

Potom platí:

$$ \tilde{N}(s) = \tilde{G_0}(s) \cdot \mathcal{L}[\rho(t) N(t)](s) + \dfrac{N_0}{s}. $$

Teď už pouze zbývá provést inverzní LT. Ale to je vopruz, takže budeme postupovat obráceně. Z vět o LT konvoluce a konstanty (0.~kapitola) je jasné, že:

$$ \tilde{N}(s) = \tilde{G_0}(s) \cdot \mathcal{L}[\rho(t) N(t)](s) + \dfrac{N_0}{s} $$
$$ \Leftrightarrow $$
$$ N(s) = G_0(s) * [\rho(t) N(t)] + N_0. $$

Po aplikaci definice konvoluce (opět 0.~kapitola) získáváme \textbf{Rovnici jednobodové kinetiky v integrálním tvaru}:

\begin{equation}
  \boxed{
  N(t) = N_0 + \int_0^t G_0(t-t') \rho(t') N(t')dt'.
  \label{integralni_kinetika}
  }
\end{equation}

To je všechno pěkný, ale musíme ještě znát obraz přenosové funkce $G_0(t)$. Zde musíme vycházet z definice $\tilde{G_0}(s)$, kterou převedeme na parciální zlomky a aplikací vět z 0.~kapitoly není obtížné nalézt původní funkci $G_0(t)$. To ale dělat nebudeme, musí nám stačit vědět, že něco takového je možné\footnote{:)}. Platí totiž:

$$ \tilde{G_0}(s) = \dfrac{\varphi(s)}{\Psi(s)} = \sum_{n=0}^m \dfrac{A_n}{s-s_n}, $$

kde $\varphi(s)$ je libovolný polynom, $\Psi(s)$ je polynom řádu $m$ a $s_n$ jsou jeho kořeny. $A_n$ jsou konstanty získané dle:

$$ A_n = \dfrac{\varphi(s_n)}{\Psi'(s_n)}. $$

Po převodu (suma lomených výrazů se zpět transformuje na sumu exponenciál) dostaneme:

\begin{equation}
  \boxed{
  G_0(t) = \sum_{n=0}^m A_n e^{s_n t}.
  \label{prenosova_funkce_reseni}
  }
\end{equation}

\small

\textbf{Kuchařka s postupem:}

Následní text je pouze matematická kuchařka, jak získat koeficienty $A_n$ a $s_n$. Definovaný vztah \eqref{prenosova_funkce} rozšíříme výrazem $\prod_{n=1}^m (s+\lambda_n)$, čímž se nám ve jmenovateli něco požere:

$$ \tilde{G_0}(s) \equiv \dfrac{\prod_{n=1}^m (s+\lambda_n)}{s \left ( \Lambda \prod_{n=1}^m (s+\lambda_n)  + \sum_{n=1}^m \beta_{\text{ef},n}\prod_{k=1, k \neq n}^m(\lambda_k + s) \right )}. $$

Jaké budou kořeny polynomu ve jmenovateli? První je jasný, tj. $s_0 = 0$. Výraz v hranaté závorce je natuty kladný $\rightarrow$ zbylé kořeny nemohou být kladné $\rightarrow$ zbylé kořeny jsou určitě záporné, tj. $s_1, s_2, ..., s_n < 0$.\\

Abychom získali konstanty $A_n$, je vhodné si polynomy upravit do lepšího tvaru. Z MAA1 víme:

$$ \Psi(s) = K \cdot \prod_{n=0}^m (s-s_n) $$

a jelikož známe první kořen $s_0 = 0$:

$$ \Psi(s) = K \cdot s \prod_{n=1}^m (s-s_n) $$

Chybí nám už pouze získat konstantu $K$. Tu získáme z řešení limity $\lim_{s \to \infty} s \cdot \tilde{G_0}(s)$. Z definice přenosové funkce \eqref{prenosova_funkce} platí:

$$ \lim_{s \to \infty} s \cdot \left [ \dfrac{1}{s \left ( \Lambda + \sum_{n=1}^m \dfrac{\beta_{\text{ef},n}}{\lambda_n + s} \right )} \right ] = \dfrac{1}{\Lambda} $$

a z námi upraveného tvaru pro polynom $\Psi(s)$ (ten, ve kterém se objevuje konstanta $K$) zase platí:

$$ \lim_{s \to \infty} s \cdot \left [ \dfrac{\prod_{n=1}^m (s+\lambda_n)}{K \cdot s \prod_{n=1}^m (s-s_n)} \right ] = \dfrac{1}{K} $$

Je tedy jasné, že:

$$ K = \Lambda, $$

a tedy:

$$ \tilde{G_0}(s) = \dfrac{\prod_{n=1}^m (s+\lambda_n)}{\Lambda \cdot s \prod_{n=1}^m (s-s_n)}. $$

Pro ty co se ztratili v tom, o co se vlastně snažíme. Pokoušíme se nalézt konstanty $A_n$, které jsme si pro připomenutí definovali jako:

$$ \tilde{G_0}(s) = \dfrac{\varphi(s)}{\Psi(s)} = \sum_{n=0}^m \dfrac{A_n}{s-s_n}, $$

$$ A_n = \dfrac{\varphi(s_n)}{\Psi'(s_n)}, $$

kde v našem značení:

$$ \Psi(s) = \Lambda \cdot s \prod_{n=1}^m (s-s_n). $$

Funkci $\Psi(s)$ musíme derivovat, což je jednoduché, jelikož se jedná o produkt. Pro jeho derivaci platí:

$$ \dfrac{d}{dx} \left ( \prod_{i=1}^k f_i(x) \right ) = \sum_{i=1}^k \left ( \dfrac{f_i'(x)}{f_i(x)} \prod_{j=1,j \neq i}^k f_j(x) \right ). $$

Už je pozdě a je zbytečné rozepisovat postupy, zkrátka platí, že:

$$ \Psi'(s) = \Lambda \prod_{n=1}^m (s-s_n) + \Lambda \cdot s \sum_{n=1}^m \prod_{k=1, k \neq n}^m (s-s_k). $$

Při dosazení prvního kořene $s_0 = 0$ je řešení triviální, při dosazování zbylých kořenů je první člen vždy nulový a u druhého členu vypadne suma. Tedy:

$$ \Psi'(s_0) = \Lambda \cdot \prod_{i=1, i \neq n}^m (-s_i), \: \: \: n = 0, $$

$$ \Psi'(s_n) = \Lambda \cdot s_n \prod_{i=1, i \neq n}^m (s_n-s_i), \: \: \: \forall \: n \in \widehat{m}. $$

\normalsize

Rovněž platí (a bude se hodit):

$$ \sum_{n=0}^m A_n = \dfrac{1}{\Lambda}. $$

\subsubsection{Analytické řešení s vlivem zpožděných neutronů}

Vycházíme z řešení rovnice jednobodové kinetiky v integrálním tvaru \eqref{integralni_kinetika}. Rovnice je analyticky řešitelná pouze pro 3 různé případy $\rho$:

\begin{itemize}
  \item impuls $\rightarrow$ impulsní charakteristika,
  \item konstanta $\rightarrow$ přechodová charakteristika,
  \item lineární/periodická závislost.
\end{itemize}

\textbf{a) Impulsní charakeristika}

Jde tedy o případ, kdy do kritického systému zasáhneme kladnou/zápornou reaktivitu ve tvaru impulsu. Reálně nemůže nastat, ale matematicko-idealisticky lze $\rho$ vyjádřit pomocí Dirackovy $\delta$ funkce:

$$ \rho(t) = \rho_0 \delta (t). $$

Dále tedy řešíme rovnici \eqref{integralni_kinetika}:

$$ N(t) = N_0 + \int_0^t G_0(t-t') \rho(t') N(t')dt', $$

$$ N(t) = N_0 + G_0(t) \rho_0 N_0. $$

Nás ale zajímá relativní změna $\dfrac{\Delta N}{N_0}$, přičemž za $G_0(t)$ dosadíme z \eqref{prenosova_funkce_reseni}:

\begin{equation}
  \boxed{
  \dfrac{\Delta N}{N_0} = \rho_0 G_0 (t) = \rho_0 \sum_{n=0}^m A_n e^{s_n t}.
  \label{impulsni_charakteristika}}
\end{equation}

Jelikož jsou členy $s_n < 0 \: \: \: \forall \: n \in \widehat{m}$ a $s_0 = 0$, jsou exponenciály ve výrazu \eqref{impulsni_charakteristika} klesající, tudíž i průběh relativní změny četnosti je klesající. Nejde ovšem k nule a v $t=0$ nejde k nekonečnu! Dále by nás tedy mohly zajímat limitní případy pro $t \rightarrow 0^+$ a $t \rightarrow \infty$. Řešíme přes limity:

$$ \lim_{t \to 0^+} \dfrac{\Delta N}{N_0} = \rho_0 \sum_{n=0}^m A_n = \dfrac{\rho_0}{\Lambda} \doteq \dfrac{\rho_0}{\ell}, $$

$$ \lim_{t \to \infty} \dfrac{\Delta N}{N_0} = \rho_0 \lim_{s \rightarrow 0} s \tilde{G_0}(s) = \dfrac{\rho_0}{\Lambda + \sum_{i = 1}^m \dfrac{\beta_{\text{ef},i}}{\lambda_i}} \doteq \dfrac{\rho_0}{\ell^*}. $$

Průběh relativní změny četnosti při vnesení kladné i záporné reaktivity zobrazuje obrázek \ref{fig_impulsni}.

\begin{figure}[H]
  \centering
  \includegraphics[width=0.6\textwidth]{img/impulsni.jpg}
  \caption{Závislost relativní změny četnosti neutronů pro impulsní charakteristiku.}
  \label{fig_impulsni}
\end{figure}

Je tedy jasné, že v čase $t=0$ je relativní změna četnosti neutronů ovlivněna okamžitými neutrony a v čase $t \rightarrow \infty$ zpožděnými neutrony. V reálu se u vysokých vložených reaktivit (TRIGA) výrazně projevují zpětné vazby. Výrazy platí pro záporné i kladné vnesené reaktivity.\\

\textbf{b) Přechodová charakteristika}

Jde o případ, kdy je vložená reaktivita konstantní, tedy:

$$\rho(t) = \rho_0$$

Lze aplikovat i v reálu (pád tyče), i když ve skutečnosti to tak není (tyč padá nějakou tu dobu). Pro řešení vycházíme ještě z doby před zavedením integrální podoby jednobodové kinetiky\footnote{Tady si pro další usnadnění života přepíšu proměnou v LT pomocí $z$, aby se mi nepletlo s obecným řešením.}:

$$ \tilde{N}(z) = \dfrac{N_0}{z} + \tilde{G_0}(z) \cdot \mathcal{L}[\rho(t) N(t)](z), $$

$$ \tilde{N}(z) = \dfrac{N_0}{z} + \rho_0 \tilde{G_0}(z) \tilde{N}(z).$$

Nás zajímá relativní četnost $\dfrac{\tilde{N}(z)}{N_0}$ a zároveň za $\tilde{G_0}(z)$ dosadíme z \eqref{prenosova_funkce}, tedy:

\begin{equation}
  \boxed{
  \dfrac{\tilde{N}(z)}{N_0} = \dfrac{1}{z \left ( 1-\rho_0 \tilde{G_0}(z) \right )} = \dfrac{\Lambda + \sum_{i = 1}^m \dfrac{\beta_{\text{ef},i}}{z + \lambda_i}}{z \left ( \Lambda + \sum_{i = 1}^m \dfrac{\beta_{\text{ef},i}}{z + \lambda_i} \right ) - \rho_0}.
  \label{prechodova_charakteristika_LT}}
\end{equation}

Rovnice \eqref{prechodova_charakteristika_LT} se dá transformovat úplně stejně, jako v minulé kapitole $\rightarrow$ převedu na parciální zlomky:

$$ \dfrac{\tilde{N}(z)}{N_0} = \sum_{n=0}^m \dfrac{C_n}{z-z_n}, $$

\begin{equation}
  \boxed{
  \dfrac{N(t)}{N_0} = \sum_{n=0}^m C_n e^{z_n t}.
  \label{prechodova_charakteristika}}
\end{equation}

Zaměříme-li se na znaménka koeficientů $C_n$ a $z_n$, tak:

\begin{itemize}
  \item $z_0$ má stejné znaménko jako $\rho_0$,
  \item $z_n < 0 \: \: \: \forall \: n \in \widehat{m}$,
  \item $C_0 > 0$,
  \item $C_n$ mají opačná znaménka než $\rho_0, \: \: \: \forall \: n \in \widehat{m}$.
\end{itemize}

Pro $\rho_0 < 0$ bude relativní četnost v čase exponenciálně klesat (což bude ovlivněno největším $|z_n|$, což je závislé na $\Lambda$ $\rightarrow$ s menším $\Lambda$ očekáváme strmější nástup).\\

Pro $\rho_0 > 0$ po chvíli převládne kladné $z_0$ s $C_0$ a relativní četnost exponenciálně poroste. Do tohoto zlomového okamžiku bude relativní četnost také růst, ale s jiným průběhem.

Průběh relativní četnosti při vnesení kladné i záporné reaktivity zobrazuje obrázek \ref{fig_prechodova}.

\begin{figure}[H]
  \centering
  \includegraphics[width=0.6\textwidth]{img/prechodova.jpg}
  \caption{Závislost relativní četnosti neutronů pro přechodovou charakteristiku.}
  \label{fig_prechodova}
\end{figure}

\small

\textbf{Př. 3:}

Vyzkoušíme si přechodovou charakteristiku na jedné skupině zpožděných neutronů. Tu bychom mohli získat pomocí $\bar{\lambda}$ středováním přes $\beta_{\text{ef}}$ jako:

$$ \beta_{\text{ef}} = \sum_{i = 1}^m \beta_{\text{ef},i}, $$

$$\bar{\lambda} = \dfrac{\sum_{i=1}^m \beta_{\text{ef},i} \lambda_i}{\beta_{\text{ef}}}, $$

případně pomocí středování $\bar{\tau}$:

$$ \bar{\tau} = \dfrac{\sum_{i=1}^m \beta_{\text{ef},i} \tau_i}{\beta_{\text{ef}}}. $$

Nutno poznamenat, že oba způsoby nejou navzájem ekvivalentní a nedávají stejné hodnoty. S ohledem na skupiny upřednostňuje každý delší, případně kratší skupinu.\\

Při řešení vycházíme z rovnice \eqref{prechodova_charakteristika_LT}, kterou převádíme pomocí parciálních zlomků na řešení \eqref{prechodova_charakteristika}. Máme pouze jednu skupinu zpožděnek, tudíž řešíme rovnici tvaru:

$$ \dfrac{\tilde{N}(z)}{N_0} = \dfrac{\Lambda + \dfrac{\beta_{\text{ef}}}{z + \bar{\lambda}}}{z \left ( \Lambda \dfrac{\beta_{\text{ef}}}{z + \bar{\lambda}} \right ) - \rho_0} $$

a hledáme pouze koeficienty $C_0$, $C_1$ a kořeny $z_0$, $z_1$. Pro nalezení kořenů pokládáme jmenovatele nule, což po pár řádcích úprav vede na kvadratickou rovnici:

$$ z \left ( \Lambda + \dfrac{\beta_{\text{ef}}}{z + \bar{\lambda}} \right ) - \rho_0 = 0 $$

$$ \Lambda z^2 + (\Lambda \bar{\lambda} + \beta_{\text{ef}} - \rho_0) z + (- \rho_0 \bar{\lambda}) = 0 $$

$$ z = \dfrac{\Lambda \bar{\lambda} + \beta_{\text{ef}} - \rho_0}{2 \Lambda} \left ( -1 \pm \sqrt{1 + \dfrac{4 \Lambda \rho_0 \bar{\lambda}}{(\Lambda \bar{\lambda} + \beta_{\text{ef}} - \rho_0)^2}} \right ). $$

Nyní provedeme pář předpokladů, které platí pro LWR:

\begin{itemize}
  \item $\Lambda \approx 10^{-4}$,
  \item $\bar{\lambda} \approx 10^{-1}$,
  \item $\beta_{\text{ef}} \approx 10^{-2}$
\end{itemize}

a předpoklad malých změn, tj. $\rho_0 \approx 10^{-3}$ (tedy, že $|\rho_0| << \beta_{\text{ef}}$). V takovém případě jsou členy:

\begin{itemize}
  \item $4 \Lambda \rho_0 \bar{\lambda} \approx 10^{-8}$,
  \item $\Lambda \bar{\lambda} + \beta_{\text{ef}} - \rho_0 \approx 10^{-2}$,
\end{itemize}

což znamená, že: $|4 \Lambda \rho_0 \bar{\lambda}| << |\Lambda \bar{\lambda} + \beta_{\text{ef}} - \rho_0|$. Poté se nám i výpočet kořene z kvadratické rovnice zjednoduší (protože zlomek v odmocnině je velmi malý, pro nás nulový):

$$ z = \dfrac{\Lambda \bar{\lambda} + \beta_{\text{ef}} - \rho_0}{2 \Lambda} \left ( -1 \pm 1 \right ). $$

Prvním pohledem by se mohlo zdát, že jedním z kořenů je $z_0 = 0$. To ovšem zcela očividně není pravda (vliv zjednodušení) a musíme na něj přijít jinak. Určitě ale platí druhý kořen, tj.:

$$ z_1 = \dfrac{\rho_0 - \beta_{\text{ef}} - \Lambda \bar{\lambda}}{\Lambda} \approx -\dfrac{\beta_{\text{ef}} - \rho_0}{\Lambda}. $$

 Tento kořen je jistě záporný $\rightarrow$ $z_0$ musí mít stejné znaménko jako $\rho_0$. Ten zjistíme pomocí Viétových vzorců:

$$ z_0 \cdot z_1 = \dfrac{c}{a} = \dfrac{-\rho_0 \bar{\lambda}}{\Lambda}, $$

a tedy tím pádem:

$$ z_0 = \dfrac{\rho_0 \bar{\lambda}}{\beta_{\text{ef}} - \rho_0}, $$

což má skutečně stejné znaménko jako $\rho_0$. Ještě nalezneme koeficienty $C_0$ a $C_1$. Platí (viz kuchařka někde nahoře):

$$ C_n = \dfrac{\varphi(z_n)}{\Psi'(z_n)} = \dfrac{\Lambda + \sum_{i = 1}^m \dfrac{\beta_{\text{ef},i}}{z_n + \lambda_i}}{\Lambda  + \sum_{i = 1}^m \dfrac{\beta_{\text{ef},i} \lambda_i}{(z_n + \lambda_i)^2}} = ||\text{pro 1 skupinu}|| =  \dfrac{\Lambda + \dfrac{\beta_{\text{ef}}}{z_n + \bar{\lambda}}}{\Lambda  + \dfrac{\beta_{\text{ef}} \bar{\lambda}}{(z_n + \bar{\lambda})^2}}. $$

Nalezneme nejprve $C_0$. Po dosazení:

$$ C_0 = \dfrac{\Lambda + \dfrac{\beta_{\text{ef}}}{\dfrac{\rho_0 \bar{\lambda}}{\beta_{\text{ef}} - \rho_0} + \bar{\lambda}}}{\Lambda  + \dfrac{\beta_{\text{ef}} \bar{\lambda}}{ \left ( \dfrac{\rho_0 \bar{\lambda}}{\beta_{\text{ef}} - \rho_0} + \bar{\lambda} \right ) ^2}}. $$

Zase si ulehčíme život. Při předpokladu: $\dfrac{\rho_0 \bar{\lambda}}{\beta_{\text{ef}} - \rho_0} \approx 0$ pro výraz v čitateli a pouze jednou ve jmenovateli\footnote{Proč ve jmenovateli pouze jednou? Tady mi to fakt nejde do hlavy. Dle mého by bylo správnější udělat rozvoj do 1. řádu a něco poškrtat, jenže potom by nevyšel tak hezký výsledek jako u Bédi ve skriptech...} dostaneme po pár úpravách:

$$ C_0 = \dfrac{\Lambda \bar{\lambda} + \beta_{\text{ef}}}{\Lambda \bar{\lambda} + \beta_{\text{ef}} - \rho_0}, $$

což po aplikaci dalšího předpokladu: $\Lambda \bar{\lambda} \approx 0$ dá vzniku finálnímu výsledku:

$$ C_0 = \dfrac{\beta_{\text{ef}}}{\beta_{\text{ef}} - \rho_0}. $$

Člen $C_1$ získáme obdobně, nebo si pomůžeme, jelikož známe vztah pro sumu $C_n$ (podobné odvození jako pro sumu $A_n$):

$$ \sum_{i = 1}^m C_n = 1, $$

a tedy:

$$ C_1 = -\dfrac{\rho_0}{\beta_{\text{ef}} - \rho_0}. $$

Získáme tak finální tvar:

\begin{equation}
  \boxed{
  \dfrac{N(t)}{N_0} = \dfrac{\beta_{\text{ef}}}{\beta_{\text{ef}} - \rho_0} \exp{\left ( \dfrac{\rho_0 \bar{\lambda}}{\beta_{\text{ef}} - \rho_0}t \right ) } - \dfrac{\rho_0}{\beta_{\text{ef}} - \rho_0} \exp{\left ( -\dfrac{\beta_{\text{ef}} - \rho_0}{\Lambda}t \right )}.
  \label{prechodova_charakteristika_priklad}}
\end{equation}

Pro ujasnění a okomentování výsledku. První člen vyjadřuje asymptotické chování křivky, tudíž popisuje zpožděné neutrony. Druhý člen je nejintenzivnější hned z kraje intervalu a postupně vymizí, proto popisuje okamžité neutrony.\\

Pro popis je dále důležitý člen $\dfrac{\beta_{\text{ef}}}{\beta_{\text{ef}} - \rho_0}$, který vyjadřuje hodnotu relativní četnosti neutronů, která nastane při okamžitém nárůstu výkonu.\\

\normalsize

\textbf{c) Periodické chování}
